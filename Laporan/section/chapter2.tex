\chapter{PEMBIMBING DAN BIMBINGAN}
\section{Tujuan}
Untuk membantu	 mahasiswa	 dalam	 melaksanakan	 pekerjaan	 Proyek	 diperlukan pembimbing. Selain membimbin  dalam	 pelaksanaan	 Proyek,	 dosen	 pembimbing	diharapkan	 juga	 membantu	 mahasiswa	 memecahkan	 persoalan-persoalan	 lain	 yang menghambat	pelaksanaan	Proyek.

\section{ Definisi	Pembimbing	dan	Bimbingan	}
\subsection{PEMBIMBING}
Pembimbing	 adalah	 dosen	 yang	 ditunjuk	 oleh	 Koordinator Proyek II untuk	mendampingi	dalam	 pelaksanaan	 pekerjaan	 Proyek. Kesediaan dosen	 sebagai	 pembimbing	 dibuktikan	dengan penandatanganan	 proposal	 yang	 telah	 disetujui/direvisi	 dan	 diumumkan	 oleh	
koordinator	proyek	II.

\par Daftar	calon	pembimbing	adalah	sebagai	berikut	:
\begin{table}[H]
\label{dosen}
\resizebox{\textwidth}{!}{%
\begin{tabular}{|l|l|l|l|}
\hline
\multicolumn{1}{|c|}{\textbf{KODE}} & \multicolumn{1}{c|}{\textbf{NAMA DOSEN}} & \multicolumn{1}{c|}{\textbf{NIK}} & \multicolumn{1}{c|}{\textbf{EMAIL}} \\ \hline
RHA & Roni	Habibi,	S.Kom., M.T. & 103.78.069 & roni.habibi@gmail.com \\ \hline
WIR & Woro	Isti	Rahayu,	ST.,	M.T & 105.79.081 & wistirahayu@yahoo.com \\ \hline
MNF & Mohamad	Nurkamal Fauzan,	S.T.,	M.T & 113.80.159 & m.nurkamal.f@poltekpos.ac.id \\ \hline
RMA & Rolly	Maulana Awangga,	S.T.,	M.T & 215.86.148 & rolly@awang.ga \\ \hline
HKS & M.	Harry	K	Saputra,	S.T. & 213.88.109 & putra.b13@gmail.com \\ \hline
RAN & Roni	Andarsyah,	ST.,	M.Kom. & 115.88.193 & roni.andarsyah@gmail.com \\ \hline
SFP & Syafrial	Fachri Pane,	S.T. & 213.88.110 & syafrizal.fachri@gmail.com \\ \hline
CPR & Cahyo	Prianto,	S.Pd.,	M.T & 215.84.150 & chprianto@gmail.com \\ \hline
NHH & Nisa	Hanum	Harani,	S.Kom.,	M.T & 215.89.158 & nisaharani@gmail.com \\ \hline
NRN & Rd.	Nuraini,	S.F.,	S.S.,	M.Hum. & 315.72.005 & nurainisitifathonah@gmail.com \\ \hline
YHS & M.	Yusril	Helmi	Setyawan,	S.Kom.,	M.Kom. & 113.74.163 & yusrilhelmi@yahoo.com \\ \hline
\end{tabular}
}
\end{table}



\subsection{BIMBINGAN}
\par Pelaksanaan proyek 2 dan 3 kali ini menggunakan sistem kendali mingguan. Dimana setiap hari senin pertemuan dengan pembimbing ditutup hingga 8 kali pertemuan. 1 kali pertemuan wajib ada dalam satu minggu selama rentang waktu dari hari senin hinggga senin kembali. Untuk mahasiswa

\par Dalam pelaksanaan proyek 2 kali ini menggunakan sistem penilaian yang berbeda, yaitu dengan menggunakan sistem kendali mingguan. Sistem pertemuan dengan dosen akan dilakukan dengan kebijakan sebagai berikut :
\begin{itemize}

\item Melaksanakan 1 kali pertemuan wajib dengan dosen dalam satu minggu, dalam rentang waktu dari hari senin hingga senin kembali.

\item Pertemuan akan ditutup ketika sudah mencapai 8 kali pertemuan.

\item  Setiap keterlambatan maka nilai nol baik dari dosen maupun dari mahasiswanya.

\end{itemize} 

\par Untuk penggunaan sistem pegendali mingguan, diharapkan mahasiswa mengikuti langkah-langkah berikut :

\begin{enumerate}
 \item Mengisi google form yang akan disediakan nanti
 
 \item Membuat qr code dengan menggunakan program kepo
 \par Berikut merupakan tata cara penggunaan program kepo :
 \begin{enumerate}
 	\item Clone program kepo di \url{https://github.com/awangga/kepo} (untuk tata cara clone dapat dilihat di bab IX Step 1-5).
 	
 	\item jalankan command line dan pindahkan direktori ke tempat kalian menyimpan program kepo. ( pastikan anda sudah mengistall python 3, jika belum silahkan download terlebih dahulu pada link berikut \url{https://www.python.org/downloads/}, dan melakukan instalasi )
 	
 	\item Jalankan ( pip install -r requirements.txt ) di dalam command line.
 	
 	\item Lalu jalankan ( python main.py).
 	\item Silahkan mengisi NPM dan Proyek yang sedang dikerjarkan, \\
 	 		NPM : \\
 	 		11840** \\
 	 		Proyek : \\
 	 		2
 	 		\\ Dan QRCODE berhasil dibuat, nantinya QRCODE ini lah yang nantinya akan digunakan selama melakukan bimbingan.
 \end{enumerate}
 
 \item Melakukan pull request foto pada repo foto di link berikut \url{https://github.com/D4TI/2018} \\
 (Silahkan taruh di folder kecil dengan nama file NPM.jpg (.jpg harus huruf kecil semua). Contoh 113040087.jpg, Resolusi foto 200 x 287 dengan ukuran file tidak lebih dari 50 KB. Dan foto masih tampak jelas).\\
 ( untuk tata cara melakukan pull request dapat dilihat di bab IX Step 1-9)
 
 \item Untuk melihat nilai mingguan bisa dilihat pada link berikut \url{https://docs.google.com/spreadsheets/d/1T9hgiuLQEGDhs5BUB4dXFYtj-jdIOZR0ofA-52 \\TWqAU/edit#gid=0}
 
\end{enumerate}

 \par Penilaian dilakukan oleh dosen yang bersangkutan dengan cara melakuka scan terhadap qrcode yang telah dibuat oleh mahasiswa dari program kepo, dan dosen akan melakukan login dengan menggunakan email poltekpos.\\ \textbf{Nilai yang lolos sidang minimal C.}
 
\section{Syarat Pembimbing}
Latar	belakang	Pembimbing	diharapkan	mempunyai	disiplin	ilmu	yang	sesuai	dengan	topik	
pekerjaan	Proyek II.

\section{Syarat	Bimbingan}
\subsection{Pelaksanaan Bimbingan}
\par 
Agar	 terlaksananya	 proses	 bimbingan	 dengan	 lancar,	 maka	 agar	 bisa	 melakukan	
bimbingan	harus	mengikuti	aturan	sebagai	berikut	:

\begin{enumerate}
	\item Mengontak	 pembimbing	 hanya	 melalui	 grup	 whatsapp.	 Untuk	 menguji kerjasama	di	dalam	team,	dan	mengurangi	pertanyaan	yang	sama.	Bentuk	grup whatsapp	dengan	nama	grup	“Proyek	2”,	berisi	\textbf{seluruh} rekan-rekan	mahasiswa Proyek	 2	 dari	 satu	 pembimbing. Seluruh pertanyaan dan tanggapan melalui	grup	whatsapp tersebut,	 tidak	ada	 pertanyaan	langsung ke	 pembimbing	 tanpa	
melalui	grup.
	
	\item Datang	 bimbingan	 bersama	 kelompoknya,	 jika	 3x	 bimbingan	 tidak	 bersama kelompoknya	maka	tidak	dapat	melaksanakan	bimbingan	Proyek	2.

	\item Melakukan bimbingan menggunakan \textbf{sistem kendali mingguan}
	seperti yang sudah dijelaskan diatas.
	
	\item Pertanyaan	 teknis	 pemrograman	 yang	 diajukan	 ke	 pembimbing hanya	 yang	tidak	dijawab	oleh	portal stackover	flow	selama	1x24	Jam.
	
	\item Pergantian	 judul	 atas	 persetujuan	 pembimbing	 dan	 wajib	 lapor	 kepada	
kordinator	proyek	2.

\end{enumerate}

\subsection{Jumlah Minimum Bimbingan}
Jumlah	 minimum	 bimbingan	 tidak	 dibatasi,	 akan	 tetapi	 nilai	 total	 bimbingan	 minimal agar	bisa	mengikuti	sidang	adalah	80.	Satu	minggu	untuk	penilaian	satu bimbingan,	jika pembimbing	 berhalangan	 pada	 minggu	 tersebut	 bisa	 dilakukan	 secara	 daring	 atau diakumulasikan	pada	pertemuan	minggu	selanjutnya.

\subsection{Bimbingan Tidak Sesuai Dengan Ketentuan}
Mahasiswa	 yang	 melaksanakan	 bimbingan	 tidak	 sesuai	 dengan	 ketentuan,	 tidak diijinkan	 untuk	 sidang,	 harus	 melengkapi	 nilai minimum	 bimbingan	 sebelum melaksanakan	sidang Proyek pada	lembar	bimbingan	disertakan	materai.