\chapter{CARA MERUJUK DAN MENULIS DAFTAR RUJUKAN (PUSTAKA)}
Pembuatan	daftar	pustaka \textbf{diwajibkan} menggunakan	 fitur	Reference	Bibiliography	di	Microsoft Office	sehingga	daftar	pustaka	tercipta	dengan	otomatis.\\
Cara	merujuk	daftar	pustaka	adalah	sebagai	berikut:
\begin{enumerate}
\item Daftar	Pustaka	disusun	menurut	urutan	kutipan	dan	diberi nomor	urut	mulai	dari	[1].
\item Judul	buku	tidak	boleh	disingkat.
\item Penyingkatan	kependekan	Jurnal	Ilmiah	harus	mengikuti	yang	telah	lazim	dilakukan.
\item Nama	keluarga	(nama	belakang)	ditulis	terlebih	dahulu,	diikuti	dengan	singkatan	nama	depan.
\item Semua	nama	pengarang	harus	ditulis	sesuai	dengan	urutannya	di		dalam	artikel	/	buku.
\end{enumerate}

\par Penjelasan	lebih	rinci	mengenai	cara	merujuk	dan	menulis	daftar	rujukan	dijelaskan	sebagai	
berikut

\section{Cara	Merujuk}
Perujukan	 dilakukan	 dengan	menggunakan	 nama	 	 akhir	 dan	 tahun	 di	 antara	 tanda	 kurung. Jika	ada	dua	penulis,	perujukan	dilakukan	dengan	cara	menyebut	nama	akhir	kedua	penulis	
tersebut.	Jika	penulisnya	lebih	dari	dua	orang,	penulis	rujukan	dilakukan	dengan	cara	penulis	nama	pertama	dari	penulis	 tersebut	diikuti	dengan \textit{dkk}. Jika	nama	penulis	 tidak	disebutkan,	yang	 dicantumkan	 dalam	 rujukan	 adalah	 nama	lembaga	 yang	menerbitkan,	 nama	 dokumen yang diterbitkan,	 atau	 nama	 koran.	 Untuk	 karya	 terjemahan,	 perujukan	 dilakukan	 dengan	cara	menyebutkan	nama	penulis	aslinya.	Rujukan	dari	dua	sumber	atau lebih	yang	ditulis	oleh penulis	yang	berbeda	dicantumkan	dalam	satu	tanda	kurung	dengan	titik	koma	sebagai	tanda	pemisahnya.

\section{Cara	Merujuk	Kutipan	Langsung}
\subsection{Kutipan	Kurang	dari	40	Kata}
Kutipan	 yang	 berisi	 kurang	 dari	 40	 kata	 ditulis	 di	 antara	 tanda	 kutip	 (“…”)	 sebagai	bagian	 yang	 terpadu	 dalam	 teks	 utama,	 dan	 diikuti	 nama	 penulis,	 tahun	 dan	 nomor	halaman.	 Nama	 penulis	 dapat	 ditulis	 secara	 terpadu	 dalam	 teks	 atau	 menjadi	 satu	dengan	tahun	dan	nomor	halaman	di	dalam	kurung.	Lihat	contoh	berikut. Nama	penulis disebut	dalam	teks	secara	terpadu \\
\\

\textbf{Contoh :}\\
Tersine	 (1994:	 28)	 menyatakan	 “tekanan	 pasar	 memaksa	 organisasi	 untuk	menghasilkan	produk	yang	lebih	beragam	dan	kemampuan	pengiriman	yang	lebih	baik” Nama	penulis	disebut	bersama	dengan	tahun	penerbitan	dan	nomor	halaman.
\\
\\

\textbf{Contoh :}\\
Hal	 tersebut	 berdasarkan	 pada	 pernyataan	 “tekanan	 pasar	memaksa	 organisasi	 untuk	menghasilkan	produk	yang	lebih	beragam	dan	kemampuan	pengiriman	yang	lebih	baik”	
(Tersine,	1994:28). \\
Jika	ada	tanda	kutip	dalam	kutipan,	digunakan	tanda kutip	tunggal	(‘…’).
\\
\\

\textbf{Contoh :}\\
Ini	sejalan	dengan	pernyataan	Bickelhaupt	yang	menyatakan	“Kontrak	asuransi	bersifat	pribadi	 (personal)	 dang	 ‘mengikuti’	 pribadi	 itu,	 bukan	 ‘mengikuti’	 harta	 yang	diasuransikan.”

\subsection{Kutipan	40	Kata	atau	Lebih}
Kutipan	yang	berisi	40	kata	atau	lebih	ditulis	tanpa	tanda	kutip	secara	terpisah	dari	teks	yang	mendahului,	 ditulis	 1,2	 cm	 dari	 garis	 tepi	 sebelah	 kiri	 kanan,	 dan	 diketik	 dengan	spasi	tunggal.	Nomor	halaman	juga	harus	ditulis.
\\
\\
\\
\textbf{Contoh :}\\
Harrington	 (1999	 :	 384)	 menarik	 kesimpulan	 sebagai	 berikut.“Making	 manufacturers	strictly	liable	for	all	consumer	losses	can	improve	safety	incentives	when	consumers	are	
uninformed	 about	 product	 risk,	 because	 strict	 liability	 gives	 manufacturers	 proper	incentives	 to	make	safe	products	and	induces	consumers	 to	purchase	 the	right	amount	of	risky	products.” \\
Jika	dalam	kutipan	 terdapat	paragraf	baru	lagi,	garis	barunya	dimulai	1,2	cm	dari	 tepi	
kiri	garis	teks	kutipan.

\subsection{Kutipan	yang	Sebagian	Dihilangkan}
Apabila	 dalam	 mengutip	 langsung	 ada	 kata-kata	 dalam	 kalimat	 yang	 dibuang,	 maka	kata-kata	yang	dibuang	diganti	dengan	tiga	titik. \\
\\

\textbf{Contoh :}\\
"Asuransi	konstruksi	menjamin	kerugian	akibat	kerusakan	fisik	pada	proyek	pekerjaan	teknik	sipil	…	disebabkan	kecelakaan	yang	terjadi	pada	masa	pembangunan.” \\

Apabila	ada	 kalimat	 yang	 	 dibuang,	maka	 kalimat	 yang	 dibuang	 diganti	 dengan	empat	titik.\\
\\

\textbf{Contoh :}\\
“Kerugian	 tidak	langsung	juga	 timbul	pada	bangunan	 yang	 tidak	memenuhi	 ketentuan	sehingga	harus	dilakukan penggantian	semua	atau	sebagian	bangunan	tersebut	….Maka	kerugian	 tak	 langsung	 ada	 berupa	 biaya	 membuka	 bagian	 yang	 tidak	 salah,	 nilai	 dari	bagian	yang	tidak	dirusakkan,	dan	perbedaan	nilai	bangunan	setelah	diperbaiki	dengan	nilai	bangunan	sebelumnya”	(Darmawi,	2000:144).

\subsection{Cara	Merujuk	Kutipan	Tidak	Langsung}
Kutipan	 yang	 disebutsecara	 tak	 langsung	 atau	 dikemukakan	 dengan	 bahasa	 penulis	 sendiri	ditulis	tanpa	tanda	kutip	dan	terpadu	dalam	teks.	Nama	penulis	bahan	kutipan	dapat	disebut	terpadu	 dalam	 teks,	 atau	 disebut	 dalam	 kurung	 bersama	 tahun	 penerbitannya.	 Jika	memungkinkan	nomor	halaman	disebut	terpadu	dalam	teks.
\\
\\
\textbf{Contoh :} \\
Skipper	 (1999:453)	 hanya	 melakukan	 peramalan	 permintaan	 dengan	 pendekatan	 regresi	linier.\\ Nama	penulis	disebut	dalam	kurung	bersama	tahun	penerbitannya.\\
\textbf{Contoh :}\\
Untuk	kasus	tersebut,	regresi	logistik	ternyata	memberikan	hasil	yang	lebih	baik	(Wolff,	2000	
:	144).

\subsection{Cara	Menulis	Daftar	Rujukan (Pustaka)}
Daftar	rujukan	merupakan	daftar	yang	berisi	buku,	makalah,	artikel,	atau	bahan	lainnya	yang	dikutip	 baik	 secara	 langsung	 maupun tidak	 langsung.	 Bahan-bahan	 yang	 dibaca	 akantetapi	tidak	dikutip	\textit{tidak	dicantumkan} 	dalam	daftar	rujukan,	sedangkan	semua	bahan	yang	dikutip	secara	langsung	ataupun	 tak	langsung	dalam	 teks \textit{harus} dicantumkan	dalam	daftar	 rujukan.	

Pada	 dasarnya,	 unsur	 yang	 ditulis	 dalam	 daftar	 rujukan	 secara	 berturut-turut	 	meliputi	 (1)	nama	penulis	ditulis	dengan	urutan	:	nama	akhir,	nama	awal,	dan	nama	 tengah,	 tanpa	gelar	akademik,	 (2)	 tahun	 penerbitan,	 (3)	judul,	 termasuk	 anak	judul	(\textit{subjudul}),(4)	 kota	 tempat	
penebitan,	 dan	 (5)	 nama	 penerbit.	 Unsur-unsur	 tersebut	 dapat	 bervariasi	 tergantung	 jenis	sumber	 pustakanya.	 Jika	 penulisnya	 lebih	 dari	 satu,	 cara	 penulisan	 namanya	 sama	 dengan	penulis	pertama	(Lampiran-8). \\

Nama	 penulis	 yang	 terdiri	 dari	 dua	 bagian	 ditulis	 dengan	 urutan:	 nama	 akhir	 diikuti	 koma,	nama	 awal	 (disingkat	 atau	 tidak	 disingkat	 tetapi	 harus	 konsisten	 dalam	 satu	 karya	ilmiah), diakhiri	dengan	 titik.	Apabila	sumber	yang	dirujuk	ditulis	oleh	lain,	semua	nama	penulisnya	harus	dicantumkan	dalam	daftar	rujukan.
\\
\\

\textbf{Rujukan dari Buku} \par


Tahun	 penerbitan	 ditulis	 setelah	 nama	 penulis,	 diakhiri	 dengan	 titik.	 Judul	 buku	 ditulis	dengan	 huruf	 miring,	 dengan	 huruf	 besar	 pada	 awal	 setiap	 kata,	 kecuali	 kata	
hubung.Tempat	penerbitan	dan	nama	penerbit	dipisahkan	dengan	titik	dua	(:). \\
\\
\textbf{Contoh :} \\
Magee,	J.	F.	\&	Boodman,	D.	M.	1967. \textit{Production	Planning	and	Inventory	Control.}	New	York: McGraw-Hill. \\
Jika	 ada	 beberapa	 buku	 yang	 dijadikan	 sumber	 ditulis	 oleh	 orang	 yang	 sama	 dan	diterbitkan	dalam	tahun	yang	sama	pula,	data	tahun	penerbitan	diikuti	oleh	lambang	a,	b,	
dan	 c,	 dan	 seterusnya	 yang	 urutunnya	 ditentukan	 secara	 kronologis	 atau	 berdasarkan	abjad	judul	buku-bukunya. \\
\\

\textbf{Contoh :} \\
Cummins,	J.	D.	1992a. \textit{Should	Automobile	Insurance	be	Compulsary?}	Cincinnati,	OH:	General	Publisher. \\
Cummins,	 J.	 D.	 1992b.	\textit{Should	 Automobile	 Insurance	 be	 Compulsary:	 The	 Second	Perspective}. Cincinnati,	OH:	General	Publisher.
\\
\\
\textbf{Rujukan dari Buku yang Berisi Kumpulan Artikel (Ada Editornya)}


Seperti	menulis	rujukan	dari	buku	ditambah	dengan	tulisan	(Ed.)	jika ada	satu	editor	dan	(Eds.)	jika	editornya	lebih	dari	satu,	di	antara	nama	penulis	dan	tahun	penerbitan. \\ \\

\textbf{Contoh :}
Park,	S.	\&	Browse,	R.	(Eds.).	1998. \textit{A	Text	on	Marine	Insurance}.	New	York:	Pogue. \\
Dijkstra	(Ed.).	1990. \textit{Logistics Management}. New	York:	The	Foundation	Presss.
\\
\\
\textbf{Rujukan dari Artikel dalam Buku Kumpulan Artikel (Ada Editornya)}


Nama	penulis	artikel	ditulis	di	depan	diikuti	dengan	tahun	penerbitan.	Judul	artikel	ditulis	tanpa	 cetak	 miring.	 Nama	 editor	 ditulis	 seperti	 menulis	 nama	 biasa,	 diberi	 keterangan	(Ed.)	bila	hanya	satu	editor,	dan	(Eds.)	bila	lebih	dari	satu	editor.	Judul	buku	kumpulannya	ditulis	dengan	huruf \textit{miring}, dan	nomor	halamannya	disebutkan	dalam	kurung.\\
\\
\textbf{Contoh :}\\
Hartley,	 J.T.,	Harker,	 J.O.	\&	Walsh,	D.A.	1980. Contemporary	 Issues	and	New	Directions	in	Adult	 Development	 of	 Learning	 and	Memory.	 Dalam	 L.W.	 Poon	 (Ed.) \textit{Aging	in	 the	 1980s:	Psychological	Issues}(hlm.	239-252).	Washington,	D.C.:	American	Psycological	Association.\\
 Hasan,	 M.Z.	 1990.	 Karakteristik	 Penelitian	 Kualitatif.	 Dalam	 Aminuddin	 (Ed.),\\ \textit{Pengembangan	Penelitian	Kualitatif	dalam	Bidang	Bahasa	dan	Sastra} (hlm.	12-25).	Malang:	HISKI	Komisariat	Malang	dan	YA3.
 \\
 \\
 \textbf{Rujukan dari Artikel dalam Jurnal}\\
 Nama	 penulis	 ditulis	 paling	 depan	 diikuti	 dengan	 tahun	 dan	 judul	 artikel	 yang	 ditulis	dengan	 cetak	 biasa,	 dan	 huruf	 besar	 pada	 setiap	 awal	 kata.	 Nama	 jurnal	 ditulis	 dengan	cetak	miring,	 dan	 huruf	awal	 dari	 setiap	 katanya	 ditulis	 dengan	 huruf	 besar	 kecuali	 kata	hubung.	Bagian	akhir	berturut-turut		ditulis		jurnal	tahun	keberapa,	nomor	berapa	(dalam	kurung),	dan	nomor	halaman	dari	artikel	tersebut.\\
\\
\textbf{Contoh :}\\
Wuhrer,	 J.	 1975.	Better	Group	 Corporate	Health	 Financing.	\textit{Journal	 of	Risk	 and	 Insurance}, 1(3):	47-50.
\\
\\
\textbf{Rujukan dari Artikel dalam Jurnal dari CD-ROM}
Nama	penulis	ditulis	paling	depan,	diikuti	oleh	 tanggal,	bulan,	dan	 tahun	(jika	ada).	 Judul	artikel	ditulis	dengan	cetak	biasa,	dan	huruf	besar	pada	setiap	huruf	awal	kata,	kecuali	kata	
sambung.	Nama	majalah	ditulis	dengan	huruf	kecil	kecuali	huruf	pertama setiap	kata,	dan	dicetak \textit{miring}. Nomor	halaman	disebut	pada	bagian	akhir.
\\
\\Suryana,	 1996.	 Optimalisasi	 Waktu	 Perjalanan	 dengan	 Translasi	 Nonlinier. \textit{Jurnal	Transportasi},	3(3):55-59.
\\
\\
\textbf{Rujukan dari Koran Tanpa Penulis}
\\
Nama		koran	ditulis	di	bagian	awal.	Tanggal,	bulan,	dan	tahun	ditulis	setelah	nama	koran,	
kemudian	judul	ditulis	dengan	huruf	besar-kecil	dicetak	miring	dan	diikuti	dengan	nomor	halaman.
\\
\\
\textbf{Contoh :}
Suara	Pembaruan.	26	Juni,	1998. \textit{Asuransi	Perjalanan	Wisata},	5.	
\\
\\
\textbf{Rujukan dari Lembaga yang Ditulis Atas Nama Lembaga Tersebut}
\\
Nama	 lembaga	 penanggung	 jawab	 langsung	 ditulis	 di	 depan,	 diikuti	 dengan	 tahun,	 judul	karangan	 yang	 dicetak	 miring,	 nama	 tempat	 penerbitan,	 dan	 nama	 lembaga	 yang	bertanggung	jawab	atas	penerbitan	karangan	tersebut.
\\
\\
\textbf{Contoh :}
\\
Dewan	Asuransi	Indonesia.	1989. \textit{Perkembangan	Bisnis	Asuransi	di	Indonesia	Periode}	1975-1985.	Jakarta:	Dewan	Asuransi	Indonesia.
\\
\\
\textbf{Rujukan Berupa Karya Terjemahan}\\
Nama	 penulis	asli	 ditulis	 di	 depan,	 diikuti	 tahun	 penerbitan	 karya	asli,	judul	 terjemahan,	nama	 penerjemah,	 tahun	 terjemahan,	 nama	 tempat	 penerbitan	 dan	 nama penerbit	terjemahan,.	 Apabila	 tahun	 penerbitan	 buku	 asli	 tidak	 dicantumkan,	 ditulis	 dengan	 kata \textit{Tanpa	tahun}.
\\
\\
\textbf{Contoh :} \\
Zaelani,	 G.	 1989. \textit{Suatu	 Tinjauan	 tentang	 Ukuran-ukuran	 Efisiensi	 Sistem	 Transportasi}.	Skripsi	tidak	diterbitkan.	Jakarta:	Universitas	Trisakti.
\\
\\
\textbf{Rujukan Berupa Makalah yang Disajikan dalam Seminar, Penataran, atau Lokakarya}
\\
Nama penulis ditulis paling depan, dilanjutkan dengan tahun, judul makalah ditulis dengan cetak miring, kemudian diikuti pernyataan “Makalah disajikan dalam ..”., nama pertemuan, lembaga penyelengara, tempat penyelenggaraan, dan tanggal serta bulannya.
\\
\\
\textbf{Contoh :}
\\
Huda,	 N.	 1991. \textit{Penulisan	 Laporan	 Penelitian	 untuk	 Jurnal}.	 Makalah	 disajikan	 dalam	Lokakarya	Penelitian	Tingkat	Dasar	bagi	Dosen	PTN	dan	PTS	di	Bandung,	Pusat	Penelitian	IKIP	Malang,	Malang,	12	Juli. 
\\
Karim,	Z.	1987. \textit{Tatakota	di	Negara-negara	Berkembang}. Makalah	disajikan	dalam	Seminar	Tatakota,	BAPPEDA	Jawa	Timur,	Surabaya,	1-2	September.
\\
\\
\textbf{Rujukan dari Internet berupa Karya Individual}
\\
Nama	penulis	ditulis	seperti	rujukan	dari	bahan	cetak,	diikuti	secara	berturut-turut	oleh	tahun,	 judul	 karya	 tersebut	 (\textit{dicetak	 miring})	 dengan	 diberi	 keterangan	 dalam	 kurung	(Online),	dan	diakhiri	dengan	alamat	sember	rujukan	tersebut	disertai	dengan	keterangan	kapan	di	akses,	di	antara	tanda	kurung
\\
\\
\textbf{Contoh :}
Hitchcock,	S.	Carr,	L.	\&	Hall,	W.	1996. \textit{A	Survey	of	STM	Online	Journals,	1990-95:	The	Calm	
Before	 the	 Storm}, (Online),	 \url{(http://journal.ecs.soton.ac.uk/	 survey/survey.html},	 diakses	12	Juli	1996).
\\
\\
\textbf{Rujukan dari Internet berupa Artikel dari Jurnal}
\\
Nama	penulis	ditulis	seperti	rujukan	dari	bahan	cetak,	diikuti	secara	berturut-turut	oleh	tahun,	judul	artikel,	nama	jurnal	( \textit{dicetak	miring})	dengan	diberi	keterangan	dalam	kurung	
(Online),	 volume	 dan	 nomor,	 dan	 diakhiri	 dengan	 alamat	 sumber	 rujukan	 tersebut	disertai	dengan	keterangan	kapan	diakses,	di	antara	tanda	kurung.
\\
\\
\textbf{Contoh :}\\
Griffit,	 A.I.	 1995.	 Coordinating	 Family	 and	 School:	 Mothering	 for	 Schooling.	 \textit{Education	Policy	Analysis	Archives}, (Online),	Vol.	3,	No.	1,	(\url{http://olam.ed.asu.edu/epaa/}, diakses 12 Februari	1997). \\
Kumaidi.	 1998.	 Pengukuran	 Bekal	 Awal	 Belajar	 dan	 Pengembangan	 Tesnya.\textit{ Jurnal	 Ilmu	Pendidikan}, (Online),	Jilid	5,	No.4,	(\url{http://www.jipss.ac.id},	diakses	20	Januari	2000).
\\
\\
\textbf{Rujukan dari Internet berupa Bahan Diskusi}
\\
Nama	penulis	ditulis	seperti	rujukan	dari	bahan	cetak,	diikuti	secara	berturut-turut	oleh	tahun,	 bulan,	 tahun,	 topik	 bahan	 diskusi,	 nama	 bahan	 diskusi	 (\textit{dicetak	 miring})	 dengan	diberi	 keterangan	 dalam	 kurung	 (Online),	 dan	 diakhiri	 dengan	 alamat \textit{e-mail}	 sumber	rujukan	tersebut	disertai	dengan	keterangan	kapan	diakses,	di	antara	tanda	kurung.
\\
\\
\textbf{Contoh :}
\\
Wilson,	D.	20	November	1995.	Summary	of	Citing	Internet	Sites. \textit{NETTRAIN	Discussion	List}, (Online),	(\url{NETTRAIN@UBVM.CC.BUFFALO.EDU},	diakses	22	November	1995).
\\
\\
\textbf{Rujukan	dari	Internet	berupa	E-mail	Pribadi}
Nama	 pengirim	 (jika	 ada)	 dan	 disertai	 keterangan	 dalam	 kurung	 (alamat	 e-mail	pengirim),	diikuti	secara	berturut-turut	oleh	tanggal,	bulan,	tahun,	topik	isi	bahan	(\textit{dicetak	
miring}),	 nama	 yang	 dikirimi	 disertai	 keterangan	 dalam	 kurung	 (alamat \textit{e-mail} yang	dikirim).
\\
\\
\textbf{Contoh :}
\\
Naga,	Dali	 S.	 (\url{ikip-jkt@indo.vet.id}).	 1	Oktober	 1997.	\textit{Artikel	 untuk	JIP}.	E-mail	 kepada	Ali	Saukah (\url{jippsi@mlg.ywcn.or.id}).