\chapter{PENGELOLAAN APLIKASI, SISTEM ATAU ALAT HASIL PEKERJAAN
PROYEK, JURNAL, DAN FOTO ALAT}

\section{Tujuan}
Aplikasi,	 Sistem	 atau	 Alat	 hasil	 pekerjaan	 Proyek	 sangat	 berguna	 bagi	 pengembangan	
laboratorium	 dan	 kualitas	 Proyek	 karena	 itu	 pengelolaannya	 harus	 benar-benar	
dilaksanakan.Begitu	juga	dengan	abstrak,	dan	foto	alat.

\section{Perangkat Lunak}
Pengelolaan	 alat	 atau	 perangkat	 lunak	 diserahkan	 ke	 Prodi	 D4,	 dan	 menjadi tanggung	
jawab	Ketua	Prodi.

\section{Jurnal}
Abstrak	 didokumentasikan	 oleh	 Koordinator	 Proyek	 dan	menjadi	 bahan	 penerbitan	 buku	Jurnal Proyek,	 yang	 diterbitkan	 setahun	 sekali	 oleh	 Prodi.	 Dengan	 penerbitan	 buku	
tersebut,	 diharapkan	 tidak	 ada	 terjadi	 dua	 Proyek	 dengan	 topik	 yang	 sama,	 diharapkan	juga	kualitas	Proyek	dapat	terpantau	melalui	buku	ini.

\section{Syarat Khusus}
Semua	perangkat	lunak	harus	dapat	diakses	melalui	jaringan.	Jika	Aplikasi	berbasis	:
\begin{itemize}
\item Desktop	harus	dijalankan	tanpa	debugger	(contoh:	visual	studio).
\item Mobile	harus	 dijalankan	di	perangkat	sebenarnya	bukan	emulator.
\item Web	alamat	url	tidak	boleh	terlihat	localhost.
\item Gunakan	fitur	web	service.
\item Perhatikan	masalah	keamanan	jaringan.
\end{itemize}
